\subsection{Tree Walk Interleaving}
A key bottleneck we found when we profiled code generated from tiled walks (even after vectorization) was that true dependencies between instructions were still causing
a significant number of processor stalls. Performing a walk with a single input-tree pair was not providing enough independent instructions to keep the processor busy. In order to address this \Treebeard{} applies an unroll and jam transformation on the inner most loops of the loop nest. This has 
the effect of walking multiple tree and input row pairs in an interleaved fashion. 
This mitigates the dependency stalls by enabling scheduling of instructions from independent tree walks. 

This optimization is performed in two steps. 
First a pass on the mid-level IR transforms the loop structure. 
The following listing shows the mid-level IR when the inner loop over the input rows is unrolled 
by a factor of 2 and the two resulting tree walks are jammed together.

\begin{lstlisting}[style=c++]
  forest = ensemble(...)
  for t = 0 to numTrees step 1 {
    for i = 0 to batchSize step 2 {
      tree = getTree(forest, t)
      prediction1, prediction2 = InterleavedWalk((tree, rows[i]), (tree, rows[i+1]))
    }
  }
\end{lstlisting}
Next when lowering, the operations to traverse each of the tree, input row pairs 
(the arguments to the \texttt{InterleavedWalk}) are interleaved. One step of the interleaved 
walk is listed below. 
\begin{lstlisting}[style=c++]
  // ... 
  n1 = n2 = getRoot(tree)
  // ...
  threshold1 = loadThresholds(n1)
  threshold2 = loadThresholds(n2)
  featureIndex1 = loadFeatureIndices(n1)
  featureIndex2 = loadFeatureIndices(n2)
  feature1 = rows[i][featureIndex1]
  feature2 = rows[i][featureIndex2]
  pred1 = feature1 < threshold1
  pred2 = feature2 < threshold2
  n1 = getChildNode(n1, pred1)
  n2 = getChildNode(n2, pred2)
  // ...
\end{lstlisting}

\CommentOut{
These transformations are fairly general and are not aware of the in memory representation of the model. Therefore, they 
are reusable across different in memory representations - the ones that are currently built into Treebeard or ones that 
maybe added in the future.
\TODO{AP I feel the way this section is currently written makes the optimization seem extremely trivial. Is there a different 
way to present it?}
}
\subsection{Loop Peeling and Walk Unrolling}
\Treebeard{} splits the loop that performs a tree walk into two parts. As can be made aware (through a simple pass on the IR) of the depth of the first leaf in a tree, it peels and introduces a \op{prologue} loop that walks down the tree a constant number of steps (for example, up to the depth of the first leaf) and then performs the rest of the tree walk in a separate loop.
% Next \Treebeard{} unrolls the prologue completely, avoiding all traversal induced branching in it.

Several rewrites of the peeled loop are possible. \Treebeard{} completely unrolls the prologue if the peeled loop 
walks the tree upto the depth of the first leaf. In cases where \Treebeard{} has already padded and balanced the 
tree (Section~\ref{sec:treeorder}), walk unrolling completely avoids all traversal induced spills. In the case of 
probability based tiling, the prologue loop is specialized to optimize leaf checks so that these checks 
are faster (and less general) for the most likely nodes.

% \TODO{kr: Add example?}

\subsection{Parallelization}
Currently, \Treebeard{} performs a na\"ive parallelization of the inference computation. When parallelism is enabled, the 
loop over the input rows is parallelized using a \op{parallel.for} construct in MLIR. Treebeard rewrites 
the mid-level IR by tiling the loop over the input rows with a tile size equal to the number of cores. 
As a concrete example, consider the case where we intend to perform inference using a model with 4 trees 
on a batch of 64 rows. Further, assume that we wish to parallelize this computation across 8 cores. 
\Treebeard{} then generates the following IR.
\begin{lstlisting}[style=c++]
  forest = ensemble(...)
  parallel.for i0 = 0 to 64 step 8 {
    for i1 = 0 to 8 step 1 {
      i = i0 + i1
      prediction = 0
      for t = 0 to 4 step 1 {
        tree = getTree(forest, t) 
        treePrediction = WalkDecisionTree(tree, rows[i])
        prediction = prediction + treePrediction
      }
      predictions[i] = prediction
    }
  }
\end{lstlisting}
Currently, \Treebeard{} does not perform other standard parallelization optimizations  (such as loop tiling) as they
are generic and independent of the problem domain. We leave a more thorough exploration of parallelizing decision trees
to future work.