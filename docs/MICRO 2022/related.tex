\section{Related Work}
While decision trees are heavily used there is little prior work on building 
optimizing compilers for tree-based inference. We discuss below related work and 
contrast it to \Treebeard{}.

\emph{Decision Tree Ensemble Compilers:}
Treelite\cite{Treelite} is a model compiler for decision tree ensembles and is the system
most closely related to our work. Even though Treelite compiles ensembles, it only supports 
generation of simple \op{if-else} style code. It does not perform the rich 
optimizations \Treebeard{} performs and is not easily extensible.
% It cannot perform any of the optimizations \Treebeard{} is designed to perform. 
Hummingbird\cite{Hummingbird} compiles traditional ML models to make use 
of tensor primitives so 
they can be integrated into tensor-based frameworks like TensorFlow \cite{TensorFlow}.
Hummingbird does not perform any optimizations tailored to decision trees. It 
also does not perform model specific optimizations. 

\emph{Libraries:} Currently, the most popular systems for decision tree-based models 
are libraries. XGBoost\cite{XGBoost} and LightGBM\cite{LightGBM} are the most  
popular gradient boosting libraries while scikit-learn\cite{Sklearn} is 
extremely popular for random forest models. These libraries implement both 
training and inference. However, as mentioned in Section \ref{sec:intro}, porting 
optimizations in these libraries to newer architectures and maintaining them is 
extremely difficult. Additionally, the inference routines in these libraries have 
to be general and cannot be specialized to a specific model.

Tahoe\cite{Tahoe} is a library and a performance model that performs high 
performance tree inference on GPUs. Even though it performs some model 
specific optimizations, the techniques are GPU specific and cannot easily 
be ported to CPUs. Asadi et. al.\cite{VPred} optimize tree walks
by hiding dependency stalls by interleaving tree walks. In contrast, \Treebeard{} is an extensible 
optimizing compiler that carefully implements this and many other optimizations at different levels of abstractions.
QuickScorer\cite{QuickScorer, QuickScorer1} is an algorithm that uses 
bit manipulation to compute tree predictions. Even though QuickScorer is 
extremely fast for smaller models, it does not scale well to larger 
models\cite{ProbBasedLayout}. The goal of QuickScorer is orthogonal 
to the goals of \Treebeard{} and the QuickScorer algorithm can easily  
be integrated into \Treebeard{} as another traversal strategy for the 
system to explore. Tang et. al.\cite{CacheConscious1} and Jin et. al.\cite{CacheConscious2}
build models to predict cache performance of decision tree ensembles on CPUs.
This work is again orthogonal to the work described in this paper. Some systems have been proposed 
to parallelize decision tree training on CPUs and GPUs\cite{Jansson2014gpuRFAG, Nasridinov2013DecisionTC}.


\emph{Other Systems and Techniques}: 
Ren et al \cite{PortableVM} build an intermediate language and VM to 
enable SIMD execution of decision tree inference. The SIMD execution itself is implemented 
by hand in the VM and the VM needs to be reimplemented for every supported target architecture.
Additionally, even though they perform layout optimizations, their system does not perform 
any model specific optimizations. Jo et. al.\cite{MilindTreeVectorization} describe techniques 
to vectorize tree-based applications. They do not study optimizations specific to decision trees.
Both these works vectorize tree walks by performing different tree walks on each vector lane.
The main issue with this approach is the divergence of tree walks. Another issue is that 
memory accesses are \op{gather}'s rather than vector loads. \Treebeard{}'s approach to 
vectorization solves both these issues. Also, these approaches are not precluded 
by \Treebeard{}'s tiling based vectorization. Multiple tiled tree walks can be combined
into a single vectorized walk. We leave an exploration of this to future work. 
FAST\cite{FAST} is a system that accelerates tree structured index search 
on CPUs and GPUs. FAST defines a layout for the index tree that enables vectorization 
of the tree walk. FAST uses a tree tiling approach to vectorize tree walks. However, 
FAST only uses a single triangular tile shape. \Treebeard{}'s basic tiling algorithm 
is a generalization of the tiling used in FAST. If given a perfectly balanced tree,
the basic tiling algorithm would return exactly the tiling used by FAST. 
Also, the tree walks in FAST are hand coded using intrinsics on the CPU and CUDA 
on the GPU and therefore need repeated effort to implement on each target. 

\emph{Code Generation Systems from Other Domains:}
Several compilers and code generation techniques exist for other domains.
TVM\cite{TVM}, Tiramisu\cite{Tiramisu} and Tensor 
Comprehensions\cite{TensorComprehensions} are domain specific compilers 
for deep learning models. Halide\cite{Halide} is a DSL and compiler 
primarily designed for image processing applications. Several systems
that generate optimized processing routines have also been designed 
for BLAS and signal processing. BLIS\cite{BLIS} and ATLAS\cite{atlas_sc98}
are systems to instantiate high performance BLAS routines on multiple 
target architectures. CUTLASS\cite{CUTLASS} provides building blocks 
in the form of C++ templates to quickly instantiate high performance 
BLAS functions on different GPU architectures. SPIRAL\cite{SPIRAL}
is a domain specific compilers for signal processing applications
that instantiates high performance routines for linear transformations 
like FFTs and FIR filters. FFTW\cite{FFTW} is a fast fourier transform 
compiler that generates high performance FFT routines by customizing 
the routine based on FFT size and the target machine. 
However, no such frameworks for decision tree ensembles exist currently.
\Treebeard{} is a first step in this direction and unlocks several future 
optimization opportunities.
